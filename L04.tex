%%%%%%%%%%%%%%%%%%%%%%%%%%%%%%%%%%%%%%%%%%%%%%%%%%%%%%%%%%%%%%%%
% %
% Chris Pierson %
% ECE 351 Section 51 %
% Lab 3 %
% Due 2/7/23 %
%  %
% %
%%%%%%%%%%%%%%%%%%%%%%%%%%%%%%%%%%%%%%%%%%%%%%%%%%%%%%%%%%%%%%%%

% Insert an Image:
% \begin{figure}[H]
%   \centering
%   \includegraphics[width=0.8\linewidth]{picture.PNG}
%   \caption{Insert Caption Here}
% \end{figure}

% Create a New Section:
% \section{Section Name}


\documentclass[12pt]{article}

% Language and font encoding
\usepackage[english]{babel}
\usepackage[utf8x]{inputenc}
\usepackage[T1]{fontenc}
\usepackage{graphicx}
\usepackage{amsmath}
\usepackage{caption}
\usepackage{float}
\usepackage{caption}
\usepackage{subcaption}
\usepackage{rotating}
\usepackage{setspace}

% Sets page size and margins
\usepackage[a4paper,top=3cm,bottom=2cm,left=3cm,right=3cm,marginparwidth=1.75cm]{geometry}

% Useful packages
\usepackage[colorinlistoftodos]{todonotes}
\usepackage[colorlinks=true, allcolors=blue]{hyperref}
\usepackage{listings}
\usepackage{gensymb}

%Line Spacing
\setstretch{1.5}

%-------------------Begin Editing Here---------------------

\title{%
	\textbf{LAB 4} \\
	System Step Response Using Convolution \\
	 ECE 351 }

\author{\\
	Christopher Pierson}
\begin{document}
	
%\maketitle
	
%Make a Title Page
\vspace{\fill}
\maketitle
\vspace{\fill}
\clearpage

\tableofcontents

\newpage
\section{Introduction}

In this lab, we will learn how to compute a system's step response using convolution.

%Introduction
\section{Part 1}
\subsection{Equations}

\[h_{1}(t)=e^{-2t}[u(t)-u(t-3)]\]
\[h_{2}(t)=u(t-2)-u(t-6)\]
\[h_3(t)=cos(\omega_{0}t)u(t)\]


\subsection{Methodology}

The code for the convolution was taken from the previous lab.  The functions for the hand calculations were added and plotted for comparison.

\subsection*{Code}
\begin{scriptsize}
\begin{lstlisting}
#################################################################
##  PART 2 ##

def my_conv(func_1, func_2):
    Nf1 = len(func_1)  #length of input function 1
    Nf2 = len(func_2)  #length of input function 2
    f1Extended = np.append(func_1, np.zeros((1, Nf2-1))) #makes functions same
    f2Extended = np.append(func_2, np.zeros((1, Nf1-1))) #length by adding 0s   
        
    result = np.zeros(f1Extended.shape) #output array

    for i in range(Nf2 + Nf1 - 2):  #loop through all values of both functions
        result[i] = 0
        for j in range(Nf1):  #loop through values of first function
            if(i - j + 1) > 0:
                try:  #attempts to multiply functions and set result to output
                    result[i] += f1Extended[j] * f2Extended[i - j + 1]
                except: #prints array location if there is an exception
                    print(i,j)                    
    return result    
print('length=',2*t[len(t)-1])

t_conv = np.arange(-20,2*t[len(t)-1]+steps,steps)

h1_u = my_conv(h1,uu)
h2_u = my_conv(h2,uu)
h3_u = my_conv(h3,uu)

# Code for plots
plot2 = plt.figure(figsize =(12 ,8)) # start a new figure , with
# a custom figure size
plt.subplot(3 ,1 ,1) # subplot 1: subplot format (row , column , number )
plt.plot(t_conv, h1_u) # choosing plot variables for x and y axes
plt.title('Lab 4 Convolution Plots') # title for entire figure

# (all three subplots )
plt.ylabel('h1(t)*u(t)') # label for subplot 1
plt.grid(True) # show grid on plot

plt.subplot(3, 1, 2) # subplot 2
plt.plot(t_conv, h2_u)

plt.ylabel('h2(t)*u(t)') # label for subplot 2
plt.grid(which ='both') # use major and minor grids


plt.subplot(3, 1, 3) # subplot 3
plt.plot(t_conv, h3_u)
plt.ylabel('h3(t)*u(t)') # label for subplot 2
plt.grid(which ='both') # use major and minor grids


h1 = (1/2*(1-np.exp(-2*t_conv))*u(t_conv)-1/2*(1-np.exp(-2*(t_conv-6)))*u(t_conv-6))
h2 = (t_conv-2)*u(t_conv-2) - ((t_conv-6)*u(t_conv-6))
h3 = (1/w0)*np.sin(w0*t_conv)*u(t_conv)

# Code for plots
plot2 = plt.figure(figsize =(12 ,8)) # start a new figure , with
                                     # a custom figure size
plt.subplot(3 ,1 ,1) # subplot 1: subplot format (row , column , number )
plt.plot(t_conv, h1) # choosing plot variables for x and y axes
plt.title('Lab 4 Hand Convolution Plots') # title for entire figure

# (all three subplots )
plt.ylabel('h1(t)*u(t)') # label for subplot 1
plt.grid(True) # show grid on plot

plt.subplot(3, 1, 2) # subplot 2
plt.plot(t_conv, h2)

plt.ylabel('h2(t)*u(t)') # label for subplot 2
plt.grid(which ='both') # use major and minor grids


plt.subplot(3, 1, 3) # subplot 3
plt.plot(t_conv, h3)
plt.ylabel('h3(t)*u(t)') # label for subplot 2
plt.grid(which ='both') # use major and minor grids

\end{lstlisting}
\end{scriptsize}

\subsection{Results}

 \begin{figure}[H]
	   \centering
	   \includegraphics[width=0.8\linewidth]{fig1.PNG}
	   \caption{Functions $h_{1}(t)$, $h_{2}(t)$, $h_{3}(t)$}
 \end{figure}


\section{Part 2}
\subsection{Methodology}
The code for this section was taken from the previous lab.

\subsection*{Code}
\begin{scriptsize}
\begin{lstlisting}

\end{lstlisting}
\end{scriptsize}
\subsection{Hand Calculations of Step Responses of Functions}
\paragraph{}

\[h_{1}(t)*u(t)=\frac{1}{2}(1-e^{-2t})u(t)-\frac{1}{2}(1-e^{-2(t-3)})u(t-3)\]

\[h_2(t)*u(t) = (u(t-2) - u(t-6))u(t) = (t-2)u(t-2) - (t-6)u(t-6)\]
       
\[h_3(t)*u(t) = cos(w_{0}t)u(t)*u(t) =\frac{2}{\pi}sin(\frac{\pi}{2}t)u(t)\]

\subsection{Results}

 \begin{figure}[H]
	   \centering
	   \includegraphics[width=0.8\linewidth]{conv.PNG}
	   \caption{$h_{1}(t)*u(t)$, $h_{2}(t)*u(t)$, $h_{3}(t)*u(t)$}
 \end{figure}
 
  \begin{figure}[H]
 	   \centering
 	   \includegraphics[width=0.8\linewidth]{hand_conv.PNG}
 	   \caption{$h_{1}(t)*u(t)$, $h_{2}(t)*u(t)$, $h_{3}(t)*u(t)$}
  \end{figure}

\section{Questions}

1. Leave any feedback on the clarity of lab tasks, expectations, and deliverables.
\\ \\
Lab tasks and deliverables clarity was straightforward.  

\section{GitHub Link}
\url{https://github.com/pier1580}
\end{document}