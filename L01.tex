%%%%%%%%%%%%%%%%%%%%%%%%%%%%%%%%%%%%%%%%%%%%%%%%%%%%%%%%%%%%%%%%
% %
% Chris Pierson %
% ECE 351 Section 51 %
% Lab 1 %
% Due 1/24/23 %
%  %
% %
%%%%%%%%%%%%%%%%%%%%%%%%%%%%%%%%%%%%%%%%%%%%%%%%%%%%%%%%%%%%%%%%




% Insert an Image:
% \begin{figure}[H]
%   \centering
%   \includegraphics[width=0.8\linewidth]{picture.PNG}
%   \caption{Insert Caption Here}
% \end{figure}

% Create a New Section:
% \section{Section Name}




\documentclass[12pt]{article}

% Language and font encoding
\usepackage[english]{babel}
\usepackage[utf8x]{inputenc}
\usepackage[T1]{fontenc}
\usepackage{graphicx}
\usepackage{amsmath}
\usepackage{caption}
\usepackage{float}
\usepackage{caption}
\usepackage{subcaption}
\usepackage{rotating}
\usepackage{setspace}

% Sets page size and margins
\usepackage[a4paper,top=3cm,bottom=2cm,left=3cm,right=3cm,marginparwidth=1.75cm]{geometry}

% Useful packages
\usepackage[colorinlistoftodos]{todonotes}
\usepackage[colorlinks=true, allcolors=blue]{hyperref}
\usepackage{listings}
\usepackage{gensymb}

%Line Spacing
\setstretch{1.5}

%-------------------Begin Editing Here---------------------
%Info for Title Page
%\title{Buying a House}
\title{%
	\textbf{LAB 1} \\
	\small ECE 351 }

\author{\\
	Christopher Pierson}
\begin{document}
	
%\maketitle
	
%Make a Title Page
\vspace{\fill}
\maketitle
\vspace{\fill}
\clearpage

\tableofcontents

\newpage

%Introduction
\section{Part 1}

%Read the information on the following link(copy and paste this link into another tab):
%https://docs.spyder-ide.org/current/index.html.
%2. Follow the interactive tutorials (optional)
%3. Read over the Spyder keyboard shortcut cheat sheet on Canvas. If you can’t access this,
%please notify the lab instructor.
%1
%4. Open a new file in Spyder and immediately save it as LastName FirstName ECE351 Lab1.
%Note: Spyder will not run any code if the file isn’t saved first, so this is a good habit to get
%into for all future labs.

Part 1 of the lab is a general introduction to Spyder in which keyboard shortcuts are introduced and a new code file is created.

\section{Part 2}

%Introduce methods for defining variables, arrays, and matrices in python, while exploring some
%useful operations for each. Introduce Python syntax, packages, and proper formatting for plots.
%Introduce syntax for complex numbers in Python, using numpy. Introduce some simple commands
%that can make life easier while using Python and while debugging

Part 2 of the lab is an introduction to defining variables, arrays, matrices and goes over operations for them.  It also shows how to plot in Python and format properly.  Lastly, it introduces complex numbers in Python as well as some commands to use while debugging.

\section{Part 3}

Part 3 introduces pep8 coding practices.  The purpose of pep8 is to improve the readability of Python code.  This section goes over suggested pacing, tabbing, docstrings, line wrapping and comments.

\section{Part 4}

Part 4 of lab 1 introduces Latex, goes over some Latex pointers and gives sample Latex code.


\section{Questions}

1. Which course are you most excited for in your degree? Which course have you enjoyed the most so far?
\\ \\
I don't have any courses that I am excited about, most courses I really don't know what to expect going in.  The courses I have enjoyed the most have been the ones taught by Karen Frenzel and Dr. Hess.
\\ \\ 
2. Leave any feedback on the clarity of the expectations, instructions, and deliverables.
\\ \\ 
A list of the deliverable files and what should be contained in those files would be helpful.
\end{document}