%%%%%%%%%%%%%%%%%%%%%%%%%%%%%%%%%%%%%%%%%%%%%%%%%%%%%%%%%%%%%%%%
% %
% Chris Pierson %
% ECE 351 Section 51 %
% Lab 12 %
% Due 4/25/23 %
% %
% %
%%%%%%%%%%%%%%%%%%%%%%%%%%%%%%%%%%%%%%%%%%%%%%%%%%%%%%%%%%%%%%%%

% Insert an Image:
% \begin{figure}[H]
%   \centering
%   \includegraphics[width=0.8\linewidth]{picture.PNG}
%   \caption{Insert Caption Here}
% \end{figure}

% Create a New Section:
% \section{Section Name}


\documentclass[12pt]{article}

% Language and font encoding
\usepackage[english]{babel}
\usepackage[utf8x]{inputenc}
\usepackage[T1]{fontenc}
\usepackage{graphicx}
\usepackage{amsmath}
\usepackage{caption}
\usepackage{float}
\usepackage{caption}
\usepackage{subcaption}
\usepackage{rotating}
\usepackage{setspace}

% Sets page size and margins
\usepackage[a4paper,top=3cm,bottom=2cm,left=3cm,right=3cm,marginparwidth=1.75cm]{geometry}

% Useful packages
\usepackage[colorinlistoftodos]{todonotes}
\usepackage[colorlinks=true, allcolors=blue]{hyperref}
\usepackage{listings}
\usepackage{gensymb}
\usepackage{ mathrsfs }

%Line Spacing
\setstretch{1.5}

%-------------------Begin Editing Here---------------------

\title{%
	\textbf{ ECE 351 LAB 12} \\
Filter Design 
 %\\
	 }

\author{\\
	Christopher Pierson}
\begin{document}
	
%\maketitle
	
%Make a Title Page
\vspace{\fill}
\maketitle
\vspace{\fill}
\clearpage

\tableofcontents

\newpage
\section{Introduction}
\paragraph{}
For this project, we are tasked with taking sensor data from an oscilloscope reading (Figure 1) of position measurement data for an aircraft landing system and isolating the sensor signal by filtering out unwanted noise.  
\par 
We know that the data we want is in the range of 1.8 kHz to 2.0 kHz (inclusive).  Low frequency noise is being generated by the building's ventilation system and high frequency noise is being generated by a switching amplifier which shares a ground with the position measurement device.
\par 
The goals of this project are to first identify the magnitudes and frequencies of both the unwanted and needed signals, then design a circuit that will filter them.  The low frequency noise needs to be attenuated by at least -30dB, the high frequency noise by -21dB, and the position measurement signals need to be attenuated by less than -0.3dB.  Also, all noise at frequencies greater than 100kHz must be completely attenuated.    

 \begin{figure}[H]
   \centering
   \includegraphics[width=0.8\linewidth]{noisy_signal.PNG}
   \caption{Noisy Signal}
 \end{figure}

\section{Part 1}
\subsection{Methodology}
\paragraph{}
To identify the noise magnitudes and corresponding frequencies of the undesired high and low frequency noise as well as the required position measurement signals, I used my Fast Fourier Transform function on the provided noisy signal.  Next, I took the output of the Fourier transform, and used the provided make\_stem function to make stem plots of all the desired ranges.

\subsection{Code}

\begin{scriptsize}
\begin{lstlisting}
fs = 1/1e-6
def fft2(x):  # improved fast fourier transform
    N = len(x)
    X_fft = fftp.fft(x)
    X_fft_shifted = fftp.fftshift(X_fft)
    freq = np.arange(-N/2, N/2)*fs/N
    X_mag = np.abs(X_fft_shifted)/N
    X_phi = np.angle(X_fft_shifted)
    NX = len(X_mag) # gets array length
    for i in range(0,NX):
        if(np.abs(X_mag[i]) < 1e-10): # sets corresponding array element in 
            X_phi[i] = 0              # angle array to 0 if magnitude<1e-10  
    return freq, X_mag, X_phi

def make_stem(ax, x, y, color = 'k', style = 'solid', label = ' ', linewidths = 2.5, ** kwargs):
    ax.axhline(x[0], x[-1], 0, color = 'r')
    ax.vlines(x, 0, y, color = color, linestyles = style, label = label, linewidths = linewidths)
    ax.set_ylim([1.05 * y.min(), 1.05 * y.max()])
    
freq, X_mag, X_phi = fft2(sensor_sig)

fig, (ax1, ax2) = plt.subplots(2, 1, figsize = (10 , 7))

plt.subplot(2 ,1 ,1) # subplot 1: subplot format (row , column , number )
make_stem(ax1, freq, X_mag)
plt.xlim(0,450e3)
plt.ylabel('Magnitude (V)') # label for subplot 1
plt.grid(True) # show grid on plot

plt.title('Full Range') # title for entire figure 
\end{lstlisting}
\end{scriptsize}

\subsection{Results}
 \begin{figure}[H]
   \centering
   \includegraphics[width=0.8\linewidth]{1_FR.PNG}
   \caption{Stem Plot Full Range}
 \end{figure}
 

  \begin{figure}[H]
    \centering
    \includegraphics[width=0.8\linewidth]{2_LF.PNG}
    \caption{Stem Plot Low Frequencies}
  \end{figure}
  

   \begin{figure}[H]
     \centering
     \includegraphics[width=0.8\linewidth]{3_PS.PNG}
     \caption{Stem Plot Position Sensor Frequencies}
   \end{figure}
   

    \begin{figure}[H]
      \centering
      \includegraphics[width=0.8\linewidth]{4_HF.PNG}
     \caption{Stem Plot High Frequencies}
    \end{figure}

\paragraph{}
Figure 3 shows a large spike at around 30 Hz that needs to be filtered out.  Figure 4 shows eleven distinct frequencies between 1.8Khz and 2.0Khz that need to be preserved.  Figure 5 shows a variety of frequencies and magnitudes that need to be attenuated ranging from just over 2.0Khz up to around 415kHz.

\section{Part 2}
\subsection{Equations}

\[\text{Transfer Function: }H(s) = \frac{\frac{1}{RC}s}{s^{2}+\frac{1}{RC}s+\frac{1}{LC}}\]
\[\text{Transfer Function: }H(s) = \dfrac{\beta s}{s^{2}+\beta s+\omega_{0}^{2}}\]
\[\text{Center Frequency (radians): }\omega _{0} = \dfrac{1}{\sqrt{LC}}\]
\[\text{Bandwidth: } \beta = \dfrac{1}{RC}\]

\subsection{Methodology}
\paragraph{}
For this circuit, I chose to use the standard form of a bandpass circuit with the inductor and capacitor in parallel.  I started with a capacitor value of 5$\mu$F then found that the inductor value needs to be 1.4mH for the center frequency to be 1.9kHz.
\par 
Next, I calculated a resistor value of 159$\ohm$ to get a bandwidth of 200Hz.  This ended up being far too narrow so I gradually lowered the resistor value until the Bode plot of the position measurement signal range had the proper attenuation.  I ended up with a final resistor value of 39$\ohm$ after some experimentation and verification with the Bode plots.


 \begin{figure}[H]
   \centering
   \includegraphics[width=0.8\linewidth]{circuit.PNG}
   \caption{Bandpass Circuit, R = 39$\ohm$, L = 1.4mH, C = 5$\mu$F}
 \end{figure}

\section{Part 3}
\subsection{Methodology}
\paragraph{}
To generate the Bode plots of the circuit, I used used the sig.bode python function on the transfer function of the circuit designed in the previous step.

\subsection{Code}
\begin{scriptsize}
\begin{lstlisting}
R = 39           # Resistor Value (Ohms) started with 159, needed a wider bandwidth
L = 1.4e-3       # Inductor Value (Henrys) 
C = 5e-6         # Capacitor Value (Farads)
beta = 1/(R*C)
w0 = (1/(L*C))**.5

num = [beta,0]             
den = [1, beta, w0**2]

w = np.linspace(-100, 450e4, 10**5)  # start, stop, # of points in freq range
system = num, den
w, Bode_mag, Bode_phase = sig.bode(system, w)

plot0 = plt.figure(figsize =(12 ,8)) # start a new figure        

plt.subplot(211)
plt.semilogx(w/(2*np.pi), Bode_mag) 
plt.ylabel('Magnitude (dB)')
plt.title('All Frequency Bode')
plt.grid(which='both')
plt.xlim(1, 450e3)

plt.subplot(212)
plt.semilogx(w/(2*np.pi), Bode_phase)   
plt.xlabel('Hz') 
plt.ylabel('Phase (degrees)')
plt.grid(which='both')
plt.xlim(1, 450e3)

plt.show()
\end{lstlisting}
\end{scriptsize}

\subsection{Results}
 \begin{figure}[H]
   \centering
   \includegraphics[width=0.8\linewidth]{bode1.PNG}
   \caption{Bode Plot Full Range}
 \end{figure}
 

  \begin{figure}[H]
    \centering
   \includegraphics[width=0.8\linewidth]{bode2.PNG}
    \caption{Bode Plot Low Frequencies}
  \end{figure}
  

   \begin{figure}[H]
     \centering
   \includegraphics[width=0.8\linewidth]{bode3.PNG}
     \caption{Bode Plot Position Sensor Frequencies}
   \end{figure}
   

    \begin{figure}[H]
      \centering
		\includegraphics[width=0.8\linewidth]{bode4.PNG}
     \caption{Bode Plot High Frequencies}
    \end{figure}

\paragraph{}
Figure 9 shows that the position measurement signal is attenuated by less than -0.3dB.  Figures 8 and 10 show that the high and low frequencies will be attenuated by at least -21dB and -30dB respectively.

\section{Part 4}
\subsection{Methodology}
\paragraph{}
For the last part of the project, I used the sig.bilinear and sig.lfilter functions to run the noisy signal through the filter circuit designed in part 2.  Next, the filtered signal was plotted with the same ranges from part 1 with the make\_stem function to check the filter's effectiveness.

\subsection{Code}
\begin{scriptsize}
\begin{lstlisting}
z1,z2 = sig.bilinear(num,den,fs)
last1 = sig.lfilter(z1,z2,sensor_sig)

plot4 = plt.figure(figsize =(12, 8)) # start a new figure
plt.plot(t, last1)
plt.grid(True) # show grid on plot
plt.title(' Filtered Input Signal ')
plt.xlabel(' Time [s] ')
plt.ylabel(' Amplitude [V] ')

plt.show()

freq, X_mag, X_phi = fft2(last1)
fig, (ax1, ax2) = plt.subplots(2, 1, figsize = (10 , 7))

plt.subplot(2 ,1 ,1) # subplot 1: subplot format (row , column , number )
make_stem(ax1, freq, X_mag)
plt.xlim(0,450e3)
plt.ylabel('Magnitude (V)') # label for subplot 1
plt.grid(True) # show grid on plot

plt.title('Filtered Full Range') # title for entire figure 

plt.subplot(2, 1, 2) # subplot 2
make_stem(ax2, freq, X_phi)
plt.xlim(0,450e3)
plt.ylabel('Angle (radians)') # label for subplot 2
plt.grid(True) # show grid on plot
plt.xlabel('Frequency (Hz)') # label for subplot 2

plt.show()
\end{lstlisting}
\end{scriptsize}

\subsection{Results}
 \begin{figure}[H]
   \centering
   \includegraphics[width=0.8\linewidth]{filtered.PNG}
   \caption{Filtered Signal}
 \end{figure}
 
  \begin{figure}[H]
    \centering
    \includegraphics[width=0.8\linewidth]{overlaid.PNG}
    \caption{Filtered and Noisy Signals}
  \end{figure}
 
  \begin{figure}[H]
    \centering
    \includegraphics[width=0.8\linewidth]{1f.PNG}
    \caption{Filtered Full Range}
  \end{figure}
 

  \begin{figure}[H]
    \centering
   \includegraphics[width=0.8\linewidth]{2f.PNG}
    \caption{Filtered Low Frequencies}
  \end{figure}
  

   \begin{figure}[H]
     \centering
   \includegraphics[width=0.8\linewidth]{3f.PNG}
     \caption{Filtered Position Sensor Frequencies}
   \end{figure}
   

	\begin{figure}[H]
		\centering
		\includegraphics[width=0.8\linewidth]{4f.PNG}
		\caption{Filtered High Frequencies}
	\end{figure}
	
	\begin{figure}[H]
		\centering
		\includegraphics[width=0.8\linewidth]{comp1.PNG}
		\caption{All Frequency Magnitude Comparison}
	\end{figure}
	
	\begin{figure}[H]
		\centering
		\includegraphics[width=0.8\linewidth]{comp2.PNG}
		\caption{Low Frequency Magnitude Comparison}
	\end{figure}

	\begin{figure}[H]
		\centering
		\includegraphics[width=0.8\linewidth]{comp3.PNG}
		\caption{Sensor Signal Frequency Magnitude Comparison}
	\end{figure}

	\begin{figure}[H]
		\centering
		\includegraphics[width=0.8\linewidth]{comp4.PNG}
		\caption{High Frequency Magnitude Comparison}
	\end{figure}

\paragraph{}
Figures 14 and 18 show that the low frequency noise is properly attenuated.  Figures 15 and 19 show that the position measurement signals are attenuated by far less than -0.3dB.  Figures 16 and 20 show that the high frequency noise is properly attenuated.


%\newpage
\section{Questions}
\begin{enumerate}
\item 
Earlier this semester, you were asked what you personally wanted to get out of taking this
course. Do you feel like that personal goal was met? Why or why not?
\paragraph{}
I wanted to get experience using Python and I have.  I didn't care for it at first but now that I am more comfortable with it, I really like it and have even begun using it for other classes when I can.  I also have gotten a lot of experience using LaTex, which I much prefer using over Word.\\ 

\item
Please fill out the course feedback survey, I will read every word and very much appreciate
the feedback.

\item
Good luck in the rest of your education and career!

\end{enumerate}
\section{GitHub Link}
\url{https://github.com/pier1580}

%\appendix
%\section{Appendices}
%\subsection{Task 3 Output}
%\begin{scriptsize}
%\begin{lstlisting}
%Task 3 Output:
%
%Poles:  [2. 8.]
%Residues:  [ 6. -4.] 
%\end{lstlisting}
%\end{scriptsize}
%
%%\subsection{Part 2 Output}
%%\begin{scriptsize}
%%\begin{lstlisting}
%%Solution to lab H(s):
%%r= [ 1.   +0.j    -0.486+0.728j -0.486-0.728j -0.215+0.j     0.093-0.048j
%%  0.093+0.048j]
%%p= [  0. +0.j  -3. +4.j  -3. -4.j -10. +0.j  -1.+10.j  -1.-10.j]
%%k= []
%%\end{lstlisting}
%%\end{scriptsize}




\end{document}