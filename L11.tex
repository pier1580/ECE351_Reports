%%%%%%%%%%%%%%%%%%%%%%%%%%%%%%%%%%%%%%%%%%%%%%%%%%%%%%%%%%%%%%%%
% %
% Chris Pierson %
% ECE 351 Section 51 %
% Lab 10 %
% Due 4/10/23 %
% %
% %
%%%%%%%%%%%%%%%%%%%%%%%%%%%%%%%%%%%%%%%%%%%%%%%%%%%%%%%%%%%%%%%%

% Insert an Image:
% \begin{figure}[H]
%   \centering
%   \includegraphics[width=0.8\linewidth]{picture.PNG}
%   \caption{Insert Caption Here}
% \end{figure}

% Create a New Section:
% \section{Section Name}


\documentclass[12pt]{article}

% Language and font encoding
\usepackage[english]{babel}
\usepackage[utf8x]{inputenc}
\usepackage[T1]{fontenc}
\usepackage{graphicx}
\usepackage{amsmath}
\usepackage{caption}
\usepackage{float}
\usepackage{caption}
\usepackage{subcaption}
\usepackage{rotating}
\usepackage{setspace}

% Sets page size and margins
\usepackage[a4paper,top=3cm,bottom=2cm,left=3cm,right=3cm,marginparwidth=1.75cm]{geometry}

% Useful packages
\usepackage[colorinlistoftodos]{todonotes}
\usepackage[colorlinks=true, allcolors=blue]{hyperref}
\usepackage{listings}
\usepackage{gensymb}
\usepackage{ mathrsfs }

%Line Spacing
\setstretch{1.5}

%-------------------Begin Editing Here---------------------

\title{%
	\textbf{ ECE 351 LAB 11} \\
Z - Transform Operations 
 %\\
	 }

\author{\\
	Christopher Pierson}
\begin{document}
	
%\maketitle
	
%Make a Title Page
\vspace{\fill}
\maketitle
\vspace{\fill}
\clearpage

\tableofcontents

\newpage
\section{Introduction}
\paragraph{}
In this lab, we will be working with Z-transforms and python functions related to them.  We will use python to analyze a given system as well as another function that is provided in the lab handout.


\section{Part 1}

\subsection{Equations}
\paragraph{}
\[\text{Given the causal function y[k] where y[k] is the input and x[k] is the output:}\]
\[y[k] = 2x[k] − 40x[k − 1] + 10y[k − 1] − 16y[k − 2]\]
\[\text{(Assume that the system is initially at rest.)}\]


\subsection{Methodology}
%\paragraph{}
To find H(z) by hand, I first took the z transform of the given function to get:
\[Y(z)=2X(z)-40z^{-1}X(z)+10z^{-1} Y(z)-16z^{-2} Y(z)\]
Factoring Y(z) and X(z) out and rearranging yields:
\[\dfrac{Y(z)}{X(z)}=\dfrac{2-40z^{-1}}{1-10z^{-1}+16z^{-2}}\]
Putting the result in standard from:
\[H(z)=\dfrac{Y(z)}{X(z)}=\dfrac{2z^{2}-40z}{z^{2}-10z+16}\]
Factoring the denominator and saving a z for the z transform step later on:
\[\dfrac{H(z)}{z}=\dfrac{2z-40}{(z-2)(z-8)}\]
Partial Fraction Expansion:
\[\dfrac{H(z)}{z}=\dfrac{A}{(z-2)}+\dfrac{B}{(z-8)}\]

\[\left.A=\dfrac{2(z-20)}{z-8}\right|_{\;z=2}=6\]
\[\left.B=\dfrac{2(z-20)}{z-2}\right|_{\;z=8}=-4\]
\[H(z)=z \left(\dfrac{6}{z-2}-\dfrac{4}{z-8} \right)\]
Transform:
\[h[k]=6\cdot2^{k}-4\cdot8^{k}\]

\paragraph{}
The code for this lab was straightforward with no major difficulties.  The pole-zero plot code was provided in the lab handout and the other functions were followed from the python usage documentation.  I changed the frequency range by dividing by pi, and changed the angle output to degrees and centered at zero degrees with a for loop.


\subsection{Code}

\begin{scriptsize}
\begin{lstlisting}
z, p, k = zplane(num, den)
# print(z, p, k)

w, h = sig.freqz(num, den, whole= True)

w = w/np.pi  # Changes frequency range to 0 to 2pi radians
angles = np.unwrap(np.angle(h)) # code taken from python freqz() documentation

for i in range(len(angles)): # Changes output angle to degrees and centers at 0
    angles[i] = angles[i] * 180/np.pi-360

plot1 = plt.figure(figsize =(12 ,8)) # start a new figure

plt.subplot(2 ,1 ,1) # subplot 1: subplot format (row , column , number )
plt.plot(w, 20 * np.log10(abs(h)),'b') # choosing plot variables for x and y axes
plt.ylabel('H(z) Magnitude (dB)') # label for subplot 1
plt.grid(True) # show grid on plot

plt.title('Lab 11 Plots') # title for entire figure 

plt.subplot(2, 1, 2) # subplot 2
plt.plot(w, angles, 'g')
plt.ylabel('Angle (degrees)') # label for subplot 2
plt.grid(True) # show grid on plot
plt.xlabel('Frequency (radians/pi)') # label for subplot 2

plt.show()
\end{lstlisting}
\end{scriptsize}

\subsection{Results}

\begin{figure}[H]
   	\centering
   	\includegraphics[width=0.8\linewidth]{plot1.PNG}
   	\caption{pole-zero plot for H(z)}
\end{figure}
 
\begin{figure}[H]
    \centering
    \includegraphics[width=0.8\linewidth]{plot2.PNG}
   	\caption{Magnitude and phase responses of H(z)}
\end{figure}
  

%\section{Part 2}
%\subsection{Equations}
%\[x(t)=\cos(2\pi \cdot 100t)+\cos(2\pi \cdot 3024t)+\sin(2\pi \cdot 50000t)\]
%%x(t) = cos(2π · 100t) + cos(2π · 3024t) + sin(2π · 50000t)
%\subsection{Methodology}
%
%\paragraph{}
%For the second part of the lab, we plotted the given x(t) equation using the standard signal.plot function.  Next, we converted the transfer function from part I to its z-domain equivalent with the scipy.signal.bilinear() function.  Then we passed the input signal x(t) through the filter with scipy.signal.lfilter().  Lastly, we plotted the output signal.
%
%\subsection{Code}
%\begin{scriptsize}
%\begin{lstlisting}
%freq1 = 50000 # largest freq in function
%steps = 1/freq1 
%t = np.arange(0, .01, steps) 
%
%def x(t): 
%    return np.cos(2*np.pi*100*t)+np.cos(2*np.pi*3024*t)+np.sin(2*np.pi*50000*t)
%    
%plot1 = plt.figure(figsize =(12 ,8)) # start a new figure
%plt.plot(t, x(t))
%plt.grid(True) # show grid on plot
%
%plt.show()
%
%z1,z2 = sig.bilinear(num,den,freq1)
%last1 = sig.lfilter(z1,z2,x(t))
%
%plot1 = plt.figure(figsize =(12, 8)) # start a new figure
%plt.plot(t, last1)
%plt.grid(True) # show grid on plot
%
%plt.show()
%\end{lstlisting}
%\end{scriptsize}
%
%\subsection{Results}
%\begin{figure}[H]
%    \centering
%%    \includegraphics[width=0.8\linewidth]{plot4.PNG}
%   	\caption{signal x(t)}
%\end{figure} 
%
%\begin{figure}[H]
%    \centering
%%    \includegraphics[width=0.8\linewidth]{plot5.PNG}
%   	\caption{Filter Output}
%\end{figure} 

%\newpage
\section{Questions}
1. Looking at the plot generated in Task 4, is H(z) stable? Explain why or why not.
\\ \\
A stable system would have all the poles in the unit circle.  Since none of the poles are in the unit circle, this system is not stable. 
\\ \\
2. Leave any feedback on the clarity of lab tasks, expectations, and deliverables
\\ \\ 
The tasks, expectations and deliverables were all clearly stated.



\section{GitHub Link}
\url{https://github.com/pier1580}

\appendix
\section{Appendices}
\subsection{Task 3 Output}
\begin{scriptsize}
\begin{lstlisting}
Task 3 Output:

Poles:  [2. 8.]
Residues:  [ 6. -4.] 
\end{lstlisting}
\end{scriptsize}

%\subsection{Part 2 Output}
%\begin{scriptsize}
%\begin{lstlisting}
%Solution to lab H(s):
%r= [ 1.   +0.j    -0.486+0.728j -0.486-0.728j -0.215+0.j     0.093-0.048j
%  0.093+0.048j]
%p= [  0. +0.j  -3. +4.j  -3. -4.j -10. +0.j  -1.+10.j  -1.-10.j]
%k= []
%\end{lstlisting}
%\end{scriptsize}




\end{document}