%%%%%%%%%%%%%%%%%%%%%%%%%%%%%%%%%%%%%%%%%%%%%%%%%%%%%%%%%%%%%%%%
% %
% Chris Pierson %
% ECE 351 Section 51 %
% Lab 6 %
% Due 2/28/23 %
% %
% %
%%%%%%%%%%%%%%%%%%%%%%%%%%%%%%%%%%%%%%%%%%%%%%%%%%%%%%%%%%%%%%%%

% Insert an Image:
% \begin{figure}[H]
%   \centering
%   \includegraphics[width=0.8\linewidth]{picture.PNG}
%   \caption{Insert Caption Here}
% \end{figure}

% Create a New Section:
% \section{Section Name}


\documentclass[12pt]{article}

% Language and font encoding
\usepackage[english]{babel}
\usepackage[utf8x]{inputenc}
\usepackage[T1]{fontenc}
\usepackage{graphicx}
\usepackage{amsmath}
\usepackage{caption}
\usepackage{float}
\usepackage{caption}
\usepackage{subcaption}
\usepackage{rotating}
\usepackage{setspace}

% Sets page size and margins
\usepackage[a4paper,top=3cm,bottom=2cm,left=3cm,right=3cm,marginparwidth=1.75cm]{geometry}

% Useful packages
\usepackage[colorinlistoftodos]{todonotes}
\usepackage[colorlinks=true, allcolors=blue]{hyperref}
\usepackage{listings}
\usepackage{gensymb}
\usepackage{ mathrsfs }

%Line Spacing
\setstretch{1.5}

%-------------------Begin Editing Here---------------------

\title{%
	\textbf{ ECE 351 LAB 6} \\
 Partial Fraction Expansion 
 %\\
	 }

\author{\\
	Christopher Pierson}
\begin{document}
	
%\maketitle
	
%Make a Title Page
\vspace{\fill}
\maketitle
\vspace{\fill}
\clearpage

\tableofcontents

\newpage
\section{Introduction}

\paragraph{}
In this lab, we use scipy.signal.residue to perform partial fraction expansion of the differential equation from the prelab as well as a 5th order differential equation in part 2.  The Laplace domain equations of the functions will also be plotted.



%Introduction
\section{Part 1}
\subsection{Equations}

A system is described by the following differential equation:
\[y''(t)+10y'(t)+24y(t)=x''(t)+6x'(t)+12x(t)\]
The transfer function, H(s), is:
\[H(s)=\dfrac{Y(s)}{X(s)}=\dfrac{s^{2}+6s+12}{s^{2}+10s+24}\]
The time domain function y(t) for the transfer function with a step input is:

 \[y(t)=\mathscr{L}^{-1}\{H(s)\frac{1}{s}\}=\frac{1}{2}[1+2e^{-6t}-e^{-4t}]\]

\subsection{Methodology}
%To plot the step response of your prelab results and then use the scipy.signal.residue() function
%to perform partial fraction expansion on the S-domain system found in the prelab

\paragraph{}
For the first part of the lab, I plotted the step response of my prelab results and used the scipy.signal.residue() function to do a partial fraction expansion on the Laplace domain equation.

\subsection{Code}
\begin{scriptsize}
\begin{lstlisting}
#################################################################
                        ##  PART 1 ##
#################################################################
steps = 1e-5
t = np.arange(0, 2 + steps, steps) 

def prelab(t):
    return 1/2+np.exp(-6*t)-.5*np.exp(-4*t)

num0 = [1, 6, 12]
den0 = [1,10, 24]
den1 = [1,10,24,0]

tout, yout = sig.step((num0,den0), T = t)#impulse function of num & den 

#Code for plots
plot1 = plt.figure(figsize =(12 ,8)) # start a new figure

plt.subplot(2 ,1 ,1) # subplot 1: subplot format (row , column , number )
plt.plot(t, prelab(t)) # choosing plot variables for x and y axes
plt.ylabel('Hand Calculation') # label for subplot 1
plt.grid(True) # show grid on plot

plt.title('Lab 5 Part 1 Plots') # title for entire figure 

plt.subplot(2, 1, 2) # subplot 2
plt.plot(t, yout)
plt.ylabel('sig.step Output') # label for subplot 2
plt.grid(True) # show grid on plot

plt.show()

r,p,k = sig.residue(num0, den1)

print('\nSolution to prelab H(s):\nr=',r)
print('p=',p)
print('k=',k)

\end{lstlisting}
\end{scriptsize}

\subsection{Results}

 \begin{figure}[H]
   \centering
   \includegraphics[width=0.8\linewidth]{plot1.PNG}
   \caption{Part 1 Plots}
 \end{figure}




\section{Part 2}
\subsection{Equations}

%\[\text{Sine Method: }f(t)=\frac{|g|}{\omega}e^{\alpha t}sin(\omega t+\angle g)\]

The Cosine Method:

\[\text{Complex Root: }p=\alpha+j \omega\]
\[F(s)=\dfrac{F_{0}(s)}{(s+p)(s+p^{*})}\]
\[\left. k=\dfrac{F_{0}(s)}{(s-p^{*})}\right|_{s=p}=|k|\angle k\]
\[y_{c}(t)=2|k|e^{\alpha t}cos(\omega t+\angle k)u(t)\]


\[ y^{(5)}(t)+18y^{(4)}(t)+218y^{(3)}(t)+2036y^{(2)}(t)+9085y^{(1)}(t)+25250y(t)=25250x(t) \]

\subsection{Methodology}

\paragraph{}
For the second part of the lab, I repeated the method of the first part of the lab to find the r, p, and k values from the differential equation.  I used the cosine method out of the textbook to find the time domain solution of the given differential equation.  Both the cosine method and scipy.signal.step plot of the S domain functions were plotted and found to be identical.

\subsection{Code}
\begin{scriptsize}
\begin{lstlisting}
#################################################################
                        ##  PART 2 ##
#################################################################

num2 = [25250]
den2 = [1,18,218,2036,9085,25250,0]
den3 = [1,18,218,2036,9085,25250]

r,p,k = sig.residue(num2, den2)

print('\nSolution to lab H(s):\nr=',np.round(r,3))
print('p=',p)
print('k=',k)

def alpha(p):
    return p.real

def omega(p):
    return p.imag

def k_mag(r):
    return np.abs(r)

def k_ang(r):
    return np.angle(r)

def cosine_method(r,p,t): #cosine method from class
    return k_mag(r)*np.exp(alpha(p)*t)*np.cos(omega(p)*t+k_ang(r))

t=np.arange(0,4.5+steps,steps)

yout2=(cosine_method(r[0],p[0],t)+cosine_method(r[1],p[1],t)
       +cosine_method(r[2],p[2],t)+cosine_method(r[3],p[3],t)
        +cosine_method(r[4],p[4],t)+cosine_method(r[5],p[5],t))

tout, yout3 = sig.step((num2,den3), T = t)

#Code for plots
plot2 = plt.figure(figsize =(12 ,8)) # start a new figure

plt.subplot(2 ,1 ,1) # subplot 1: subplot format (row , column , number )
plt.plot(t, yout2) # choosing plot variables for x and y axes
plt.ylabel('Cosine Method') # label for subplot 1
plt.grid(True) # show grid on plot

plt.title('Lab 5 Part 2 Plots') # title for entire figure 

plt.subplot(2, 1, 2) # subplot 2
plt.plot(t, yout3)
plt.ylabel('sig.step Output') # label for subplot 2
plt.grid(True) # show grid on plot

plt.show()


\end{lstlisting}
\end{scriptsize}

\subsection{Results}


 \begin{figure}[H]
	   \centering
	   \includegraphics[width=0.9\linewidth]{plot2.PNG}
	   \caption{Part 2 Plots}
 \end{figure}


\section{Questions}
1. For a non-complex pole-residue term, you can still use the cosine method, explain why this works.
\\ \\
The cosine method still works for non-complex pole residue terms since the $\omega$ value as well as the angle of k will both be zero with non-complex terms.  This makes the cosine term drop out since the inside of the function would be zero and cosine of zero is one.  This leaves the magnitude of k multiplied by the Euler number raised to the power of alpha times t, which ends up being equivalent to the transform of a function with no complex roots.
\\ \\
2. Leave any feedback on the clarity of lab tasks, expectations, and deliverables.
\\ \\
Lab tasks and deliverables clarity was straightforward.  


\section{GitHub Link}
\url{https://github.com/pier1580}

\appendix
\section{Appendices}
\subsection{Part 1 Output}
\begin{scriptsize}
\begin{lstlisting}
Solution to prelab H(s):
r= [ 0.5 -0.5  1. ]
p= [ 0. -4. -6.]
k= []
\end{lstlisting}
\end{scriptsize}

\subsection{Part 2 Output}
\begin{scriptsize}
\begin{lstlisting}
Solution to lab H(s):
r= [ 1.   +0.j    -0.486+0.728j -0.486-0.728j -0.215+0.j     0.093-0.048j
  0.093+0.048j]
p= [  0. +0.j  -3. +4.j  -3. -4.j -10. +0.j  -1.+10.j  -1.-10.j]
k= []
\end{lstlisting}
\end{scriptsize}




\end{document}