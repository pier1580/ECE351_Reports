%%%%%%%%%%%%%%%%%%%%%%%%%%%%%%%%%%%%%%%%%%%%%%%%%%%%%%%%%%%%%%%%
% %
% Chris Pierson %
% ECE 351 Section 51 %
% Lab 5 %
% Due 2/21/23 %
% %
% %
%%%%%%%%%%%%%%%%%%%%%%%%%%%%%%%%%%%%%%%%%%%%%%%%%%%%%%%%%%%%%%%%

% Insert an Image:
% \begin{figure}[H]
%   \centering
%   \includegraphics[width=0.8\linewidth]{picture.PNG}
%   \caption{Insert Caption Here}
% \end{figure}

% Create a New Section:
% \section{Section Name}


\documentclass[12pt]{article}

% Language and font encoding
\usepackage[english]{babel}
\usepackage[utf8x]{inputenc}
\usepackage[T1]{fontenc}
\usepackage{graphicx}
\usepackage{amsmath}
\usepackage{caption}
\usepackage{float}
\usepackage{caption}
\usepackage{subcaption}
\usepackage{rotating}
\usepackage{setspace}

% Sets page size and margins
\usepackage[a4paper,top=3cm,bottom=2cm,left=3cm,right=3cm,marginparwidth=1.75cm]{geometry}

% Useful packages
\usepackage[colorinlistoftodos]{todonotes}
\usepackage[colorlinks=true, allcolors=blue]{hyperref}
\usepackage{listings}
\usepackage{gensymb}

%Line Spacing
\setstretch{1.5}

%-------------------Begin Editing Here---------------------

\title{%
	\textbf{LAB 5} \\
	Step and Impulse Response \\of an RLC Band Pass Filter \\
	 ECE 351 }

\author{\\
	Christopher Pierson}
\begin{document}
	
%\maketitle
	
%Make a Title Page
\vspace{\fill}
\maketitle
\vspace{\fill}
\clearpage

\tableofcontents

\newpage
\section{Introduction}

\paragraph{}
For this lab, we began with an RLC band-pass filter circuit (figure 1) that we converted to the Laplace domain and found the transfer function H(s).

 \begin{figure}[H]
   \centering
   \includegraphics[width=0.8\linewidth]{circuit.PNG}
   \caption{Prelab Circuit}
 \end{figure}


After finding the transfer function, we used Python to plot the time-domain response to impulse and step inputs for this circuit.


%Introduction
\section{Part 1}
\subsection{Equations}

\[\text{Transfer Function= }H(s)=\dfrac{Vout(s)}{Vin(s)}=\dfrac{(\frac{1}{RC})s}{s^{2}+(\frac{1}{RC})s+(\frac{1}{LC})}\]

\[\text{Sine Method: }f(t)=\frac{|g|}{\omega}e^{\alpha t}sin(\omega t+\angle g)\]

\[p=\dfrac{(-a+\sqrt{a^{2}-4b})}{2}=\alpha + j\omega\]

\[|g|\angle g=F'(s)|_{s=p}\]

\subsection{Methodology}

%Using the hand-solved time-domain impulse response from the prelab, implemented as a
%function.
%2. Using the scipy.signal.impulse() function with the s-domain transfer function from the
%prelab
\paragraph{}
For the first part of the lab, I setup the time domain impulse response equation using the sine method learned in class.  I used several intermediate variables for the equation, following the method learned.  
\paragraph{}
To plot the scipy.signal.impulse() function, I used the format given in the lab handout.

\subsection{Code}
\begin{scriptsize}
\begin{lstlisting}

#################################################################
                        ##  PART 1 ##
#################################################################

R = 1000
L = 27E-3
C = 100E-9

steps = 1e-5
t = np.arange(0, 1.2e-3 + steps, steps) 

num = [(1/(R*C)),0]
den = [1, (1/(R*C)), (1/(L*C))]

tout, yout = sig.impulse((num,den), T = t)#impulse function of num & den 

# def h_1(t): # numbers plugged into sine method
#     y = (10358*np.exp(-5000*t)*np.sin(18584*t+105*np.pi/180))
#     return y
# h1 = h_1(t)

def p(R,L,C):
    return (-1/(R*C)+((1/(R*C))**2-4/(L*C))**.5)/2
#print(p(R,L,C).real)

def alpha(p):
    return p(R,L,C).real

def omega(p):
    return p(R,L,C).imag

#print("ap:", alpha(p)) 

def g(R,L,C):
    return 1/(R*C)*p(R, L, C)
#print(g(R,L,C))    

def mag_g(R,L,C):
    return (g(R,L,C).real**2+g(R,L,C).imag**2)**.5                
#print("test: ",(mag_g(R,L,C)/omega(p))*np.exp(alpha(p)*.0003))

def angle_g(R,L,C):  # should return proper angle dep. on quadrant
    if g(R,L,C).real > 0 and g(R,L,C).imag > 0:
        return np.arctan(g(R,L,C).imag/g(R,L,C).real)
    elif g(R,L,C).real < 0 and g(R,L,C).imag > 0:
        return np.arctan(g(R,L,C).imag/g(R,L,C).real)+np.pi
    elif g(R,L,C).real < 0 and g(R,L,C).imag < 0:
        return np.arctan(g(R,L,C).imag/g(R,L,C).real)+np.pi
    elif g(R,L,C).real > 0 and g(R,L,C).imag < 0:
        return np.arctan(g(R,L,C).imag/g(R,L,C).real)
#print(angle_g(R,L,C))

def sine_method(R,L,C,x): #sine method from class
    return mag_g(R,L,C)/omega(p)*np.exp(alpha(p)*t)*np.sin(omega(p)*x+angle_g(R,L,C))

#print(sine_method(R,L,C,.0003))

#Code for plots
plot1 = plt.figure(figsize =(12 ,8)) # start a new figure

plt.subplot(2 ,1 ,1) # subplot 1: subplot format (row , column , number )
plt.plot(t, sine_method(R,L,C,t)) # choosing plot variables for x and y axes
plt.ylabel('Hand Calculation') # label for subplot 1
plt.grid(True) # show grid on plot

plt.title('Lab 5 Plots') # title for entire figure 

plt.subplot(2, 1, 2) # subplot 2
plt.plot(t, yout)
plt.ylabel('Scipy.Signal Output') # label for subplot 2
plt.grid(True) # show grid on plot

plt.show()

\end{lstlisting}
\end{scriptsize}

\subsection{Results}

 \begin{figure}[H]
	   \centering
	   \includegraphics[width=0.9\linewidth]{p1.PNG}
	   \caption{Part 1 Plots}
 \end{figure}


\section{Part 2}
\subsection{Methodology}

\paragraph{}
For this section, I changed the code to plot the signal impulse function from part one to plot the signal step function.  
\paragraph{}
The definition of the final value theorem was used for the equation.

\subsection{Equations}

\[\text{Final Value Theorem: }\lim_{t \to \infty}f(t)=\lim_{s \to 0}sF(s)\]


\subsection{Code}
\begin{scriptsize}
\begin{lstlisting}

#################################################################
                        ##  PART 2 ##
#################################################################

tout, yout2 = sig.step((num,den), T = t)#step response of H(s) w/ sig.step

plot2 = plt.figure(figsize =(12 ,8)) # start a new figure 

plt.subplot(2 ,1 ,1) # subplot 1: subplot format (row , column , number )
plt.plot(t,yout2) # choosing plot variables for x and y axes
plt.ylabel('Step Response of H(s)') # label for subplot 1
plt.grid(True) # show grid on plot

plt.title('Step Response of H(s)') # title for entire figure
plt.show()

\end{lstlisting}
\end{scriptsize}

\subsection{Results}

\[ \lim_{s \to 0}H(s)u(s)= \lim_{s \to 0} (\dfrac{(\frac{1}{RC})s}{s^{2}+(\frac{1}{RC})s+(\frac{1}{LC})})=0\]

 \begin{figure}[H]
	   \centering
	   \includegraphics[width=0.9\linewidth]{p2.PNG}
	   \caption{Step Response}
 \end{figure}
 
\paragraph{}
The plot of the step response looks like the plots from part one with both being the product of a sinusoid and a decaying exponential.  The step response plot for part two is different in that it starts at zero for s=0, where the part one plots started at a non zero point.  This makes sense since the step response is the convolution of the step function and the transfer function.

\section{Questions}
1. Explain the result of the Final Value Theorem from Part 2 Task 2 in terms of the physical circuit components.
\\ \\
The final value theorem shows that the function goes to 0 as s approaches 0 or as t approaches infinity for the time domain function.  In terms of the physical circuit components, this makes sense since the power source is a DC source.  When the circuit reaches steady state, the inductor will act as a wire.  Since Vout is in parallel with the inductor, Vout will go to zero.
\\ \\
2. Leave any feedback on the clarity of lab tasks, expectations, and deliverables.
\\ \\
Lab tasks and deliverables clarity was straightforward.  

\section{GitHub Link}
\url{https://github.com/pier1580}
\end{document}